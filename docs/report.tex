\documentclass[11pt,a4paper]{article}

% basic packages
\usepackage{float}
\usepackage{fullpage}
\usepackage{polski}
\usepackage{graphicx}
\usepackage[utf8x]{inputenc}

% bibliography and links
\usepackage{url}
\usepackage{cite}
\def\UrlBreaks{\do\/\do-}
\usepackage[hidelinks]{hyperref}

% tables
\usepackage{tabularx}
\usepackage{multicol}

% listings
\usepackage{listings}
\usepackage{color}
\definecolor{dkgreen}{rgb}{0,0.6,0}
\definecolor{gray}{rgb}{0.5,0.5,0.5}
\definecolor{mauve}{rgb}{0.58,0,0.82}
\lstset{
  basicstyle=\footnotesize,    
  captionpos=b,             
  commentstyle=\color{dkgreen},  
  frame=single,       
  keywordstyle=\color{blue},  
  language=Python,   
  numbers=left,     
  numbersep=7pt,   
  numberstyle=\tiny\color{gray}, 
  rulecolor=\color{black},  
  stringstyle=\color{mauve}, 
  tabsize=2,    
  title=\lstname
}

\begin{document}

\begin{titlepage}
  \begin{center}

    \textsc{\Large Politechnika Warszawska}\\[0.1cm]
    \small Wydział Elektroniki i Technik Informacyjnych
    \vfill

    \textsc{\small Zaawansowane problemy baz danych}\\[0.1cm]
    \Huge Optymalizacja danych przestrzennych pod kątem wykorzystania w~aplikacjach mobilnych\\[1.5cm]
    \small Sprawozdanie końcowe\\[2.5cm]

    \vfill

    \begin{minipage}{0.4\textwidth}
      \begin{flushleft} \large
        \emph{Autorzy:}\\[0.1cm]
        Przemysław \textsc{Barcikowski}\\
        Maciej \textsc{Suchecki}\\
      \end{flushleft}
    \end{minipage}
    \begin{minipage}{0.4\textwidth}
      \begin{flushright} \large
        \emph{Prowadzący:}\\[0.1cm]
        dr~inż.~Robert \textsc{Bembenik}\\[1cm]
      \end{flushright}
    \end{minipage}

    \vfill
    {\large \today}

  \end{center}
\end{titlepage}

\section{Treść zadania}
\paragraph{Tytuł} Optymalizacja danych przestrzennych pod kątem wykorzystania w~aplikacjach mobilnych
\paragraph{Opis}
Załadować przestrzenne dane z konkursu ,,Dane po Warszawsku'' (https://api.um.warszawa.pl) do wybranego SZPBD razem z odpowiednimi danymi nieprzestrzennymi pod kątem ich wykorzystania w aplikacjach mobilnych (optymalizacja pod kątem częstego korzystania/częstej aktualizacji danych). Przygotować zapytania przestrzenne pokazujące różne aspekty danych.

\section{Opis rozwiązania}
W~ramach zadania skupiliśmy się na~danych o~stacjach wypożyczania rowerów ,,Veturilo'' znajdujących się na~terenie Warszawy. W~celu pobrania danych przy użyciu API wspomnianego w~treści zadania został napisany skrypt w~języku Python, wykorzystujący biblioteki \emph{requests} (obsługa zapytań REST oraz formatu JSON) oraz \emph{psycopg2} (komunikacja z~bazą danych). Skrypt po~uruchomieniu próbuje połączyć się z~bazą danych. Po~uzyskaniu połączenia odpytuje on~wspomniane API tak, aby pobrać dane o~wszystkich zapisanych w bazie stacjach. Przykładowy fragment skryptu zamieszczony jest poniżej.\\

\begin{lstlisting}[caption=Fragment skryptu służącego do pobierania danych.]
def store_station_in_db(data, cursor, connection):
  object_id = data["featureMemberProperties"][0]["OBJECTID"]
  location = data["featureMemberProperties"][0]["LOKALIZACJA"]
  station_nr = data["featureMemberProperties"][0]["NR_STACJI"]
  bikes = data["featureMemberProperties"][0]["ROWERY"]
  stands = data["featureMemberProperties"][0]["STOJAKI"]
  latitude = data["featureMemberCoordinates"][0]["latitude"]
  longitude = data["featureMemberCoordinates"][0]["longitude"]
  coordinates = "POINT(%s %s)" % (latitude, longitude)
  sql = "INSERT INTO bike_stations  "
  sql += "(object_id, location, station_nr, bikes, stands, station_coordinates) "
  sql += "VALUES (%s, %s, %s, %s, %s, ST_Transform(ST_GeomFromText(%s,4326),2059));"
  cursor.execute(sql, (object_id, location, station_nr, bikes, stands, coordinates))
  connection.commit()

# load the bike stations
stations_loaded = 0
for number in range(MIN_STATION_NR, MAX_STATION_NR + 1):
  successful = False
  url = API_URL + "?id=" + MAP_ID + "&apikey=" + API_KEY
  url += "&filter=" + FILTER_PREFIX + str(number) + FILTER_SUFFIX
  while not successful:
    response = requests.get(url)
    if is_response_valid(response):
      store_station_in_db(response.json()["result"], cursor, connection)
      stations_loaded += 1
      successful = True
    else:
      print("ERROR: ", end="")
      if response.status_code == 200:
        print(response.json()["result"])
      else:
        print(str(response.status_code))
\end{lstlisting}

\newpage
Po~pobraniu każdej stacji jest ona zapisywana w~bazie danych. W~celu przechowywania danych została wykorzystana baza PostgreSQL wraz z~wtyczką PostGIS. Każda ze~stacji zawiera następujące dane:
\begin{itemize}
  \item object\_id -- identyfikator stacji (typ~\emph{INT}),
  \item location -- krótki opis dotyczący lokalizacji stacji (typ~\emph{VARCHAR(100)}),
  \item station\_nr -- numer stacji w~systemie w~zakresie 6300 - 6467 (typ~\emph{INT}),
  \item bikes -- informacja o~aktualnej liczbie rowerów znajdujących się na~stacji (typ~\emph{VARCHAR(10)} z~racji na~możliwość wystąpienia wartości tekstowej, np. ,,5+''),
  \item stands -- informacja o~dostępnej liczbie stojaków na~stacji (typ~\emph{INT}),
  \item station\_coordinates -- współrzędne geograficzne stacji (typ~\emph{GEOMETRY(POINT,2059)}).
\end{itemize}

Dane o~stacjach są~przechowywane w~tabeli \emph{bike\_stations}. W~celu optymalizacji zapytań na~dane założony został następujący indeks przestrzenny:

\begin{lstlisting}[caption=Indeks przestrzenny założony na~danych w~tabeli.]
  CREATE INDEX bike_stations_gix ON bike_stations USING GIST (station_coordinates);
\end{lstlisting}

\section{Testy}
W~celu przetestowania tak utworzonej bazy danych zostały utworzone następujące przykładowe zapytania SQL, wykonujące obliczenia na~podstawie danych przestrzennych:

% TODO wstawić zapytania

Następnie zapytania zostały wielokrotnie wykonywane odpowiednio z~założonym i~zdjętym indeksem w~celu przetestowania jego wpływu na~wydajność zapytań przestrzennych. Wyniki zostały zamieszczone w~poniższej tabeli.

% TODO tabela
%\begin{figure}[H]
%\begin{tabularx}{\textwidth}{ |p{2cm}|p{1cm}|p{5cm}|p{2.5cm}|X| }
%\hline \textbf{Adres} & \textbf{Typ} & \textbf{Parametry} & \textbf{Zwracana wartość} & \textbf{Opis} \\ \hline
%/brokers & POST & - & ID zarejestrowanego brokera & Metoda służąca do~rejestrowania nowego agenta w~symulacji. \\ \hline
%/stock/price & GET & \textbf{day} numer iteracji & cena akcji w~danej iteracji & Metoda służąca do~pobierania historycznej ceny akcji. \\ \hline
%/stock/buy & POST & \textbf{price} proponowana cena, \textbf{amount} liczba akcji do~kupienia & liczba kupionych akcji i~ich cena & Metoda służąca do~złożenia oferty zakupu akcji przez agenta. \\ \hline
%/stock/sell & POST & \textbf{price} proponowana cena, \textbf{amount} liczba akcji do~sprzedania & liczba kupionych akcji i~ich cena & Metoda służąca do~złożenia oferty sprzedaży akcji przez agenta. \\ \hline
%\end{tabularx}
%\end{figure}

\section{Wnioski}
% TODO

\end{document}
